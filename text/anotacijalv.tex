\chapter*{ANOTĀCIJA}
\noindent \textbf{Darba nosaukums:} Paaugstinātas drošības telpu piekļuves sistēmas izveide\newline
\textbf{Darba autors:} Rihards Teteris\newline
\textbf{Darba vadītājs:} Pētnieks, Mg.sc.ing. Roberts Trops\newline
\textbf{Darba apjoms:} 51 lpp., 7 attēli, 12 bibliogrāfiskās norādes, 2 pielikumi.\newline
\textbf{Atslēgas vārdi:} TELPA, DROŠĪBA, GIT, RUBY, RASPBERRYPI, RFID\newline

Šī bakalaura darba mērķis ir izveidot administrēšanas mājaslapu studentu darba telpas piekļuves sistēmai, kas piekļuves nodrošināšanai izmantos RFID kartes, kā arī automatizēt mājaslapas uzstādīšanu, izmantojot RaspberryPi kā serveri. Pašreizējā studentu darba telpas piekļuves sistēma nenodrošina pietiekamu telpas un tajā atrodošo iekārtu drošību, turklāt studentiem telpa praktiski nav pieejama jebkurā laikā, kas ir viena no prasībām.

Darbā aprakstīti izmantotie versiju vadības, infrastruktūras kā koda, testēšanas rīki un paņēmieni, kas nodrošina ātru izstrādi un ieviešanu, nezaudējot izstrādātā produkta kvalitāti. Darba ietvaros izveidotā mājaslapa ir balstīta uz Ruby on Rails tīmekļa lietojumprogrammu satvara un viss darbā uzrakstītais kods ir pārvaldīts ar Git versiju vadības sistēmu.

Izveidotā sistēma nodrošina daudz labāku piekļuves pārvaldību un žurnalēšanu. Sistēmu ieviešot tiks būtiski palielināta studentu darba telpas drošība, kā arī uzlabota piekļuve, nodrošinot piekļuvi jebkurā diennakts laikā.