\chapter{Praktiskais}

\section{RaspberryPi sagatavošana}

\section{Praktiski ar Chef}
Chef strādā pēc vedējsekotājsistēmas principa, tas nozīmē, ka vienmēr ir Chef serveris, kas kalpo kā centrmezgls un pārvalda visus uzstādītos mezglus. Atkarībā no uzņēmuma vajadzībām, Chef serveri ir iespējams uzstādīt vairākos veidos. Chef piedāvā Chef serveri kā pakalpojumu (\textit{Hosted Chef}), kuru ir iespējams izmantot par brīvu līdz 5 mezgliem. Šis ir vienkāršākais un ātrākais veids, kā praktiski sākt strādāt ar Chef, tomēr uzņēmumiem tas varētu būt neatbilstošs, jo nav pilnīga kontrole pār serveri. Ir iespējams Chef serveri uzstādīt \textit{mākonī}. Ir pieejami jau gatavi instanču attēli \textit{Amazon Webservices} un \textit{Microsoft Azure} \textit{mākoņos}, kas būtiski paātrina Chef servera uzstādīšanas laiku. Kā arī ir iespējams Chef serveri uzstādīt uz jebkura servera, uz kura uzstādīta \textit{Red Hat Enterprise Linux} vai \textit{CentOS} 5., 6. vai 7. versija, vai Ubuntu 10.04, 12.04 vai 14.04 versija.
Tā kā darbā uzstādīta tikai viena servera instance izmantojot Chef, darbā izmantots \textit{Hosted Chef} serviss. Tā arī ir paredzēts un ieteicams izmantot Chef.
Apskatīts arī, kā izmantot Chef piedāvātās automatizācijas iespējas bez Chef servera instances, tomēr nevar teikt, ka tas ir ieteicams variants.

\section{Izstrādātāja darbstacijas sagatavošana}
Veigsmīgam darbam ar Chef ir ieteicams uz darbstacijas uzstādīt šādu programmatūru:
\begin{itemize}
	\item ChefDK (\url{https://downloads.chef.io/chef-dk/})
	\item VirtualBox (\url{https://www.virtualbox.org/})
  \item Vagrant (\url{https://www.vagrantup.com/})
  \item Teksta redaktors
\end{itemize}
Lai sāktu lietot Chef ir nepieciešams uzstādīt vismaz chefDK uz darbstacijas. ChefDK sevī ietver Ruby programmēšanas valodas izpildāmos failus, kā arī bibliotēkas darbam ar Chef.
VirtualBox un Vagrant uzstādīšana nav obligāta, bet ieteicama, lai veiktu integrācijas testus uz darbstacijas izmantojot KitchenCI (\url{http://kitchen.ci/})

\subsection{Darbs ar Chef bez Chef servera}
Ir vairāki varianti, kā piemērot Chef receptes nekontaktējoties ar Chef serveri. Iespējams izmantot chef-zero vai chef-solo.
\subsubsection{Darbs ar Chef-zero}
Izmantojot \textit{chef-zero}, ir iespējams simulēt pagaidu Chef serveri datora atmiņā. Šo vēl vairāk atvieglo Knife-zero spraudnis (angl. \textit{plugin} \url{https://knife-zero.github.io/}), ar kura palīdzību var izdarīt tās pašas darbības, kā tad, ja ir uzstādīts Chef serveris.
Lai sāktu lietot Chef-zero vispirms ir nepieciešams uzstādīt chefDK (\url{https://downloads.chef.io/chef-dk/}).
Pēc tam konsolē var izpildīt sekojošas komandas:
\begin{itemize}
  \item chef gem install knife-zero
  \item chef generate repo test-repo
  \item cd test-repo
  \item chef generate cookbook cookbooks/testbook -b
  \item cd cookbooks/test-book
  \item berks vendor ../
  \item \texttt{knife bootstrap IP_ADDRESS --local-mode --ssh-user vagrant --ssh-password 'vagrant' --sudo --use-sudo-password --run-list 'recipe[testbook]' --node-name chefnode}
\end{itemize}
Karogs "-b" pie \textit{chef generate cookbook} komandas norāda, ka tiks uzstādīts Berkshelf integrācija, kas ļauj ērti atrisināt recepšu grāmatu atkarības.
Izmantojot \textit{berks vendor} komandu, Berkshelf lejuplādēs visas recepšu grāmatas, kuras ir nepieciešamas infrastruktūras uzstādīšanai.
Komanda \textit{knife bootstrap} veic vairākas darbības. Pirmkārt, tā uzstāda Chef klientu uz norādītās darbstacijas, otrkārt, tā uzstāda norādītās receptes.
Komandas \textit{knife bootstrap} karogu un mainīgo skaidrojums:
\begin{itemize}
	\item IP_ADDRESS -- pārvaldāmās darbstacijas IP adrese
	\item \texttt{--local-mode} -- norāda, ka darbs notiks ar Chef-zero
	\item \texttt{--ssh-user} -- SSH lietotāja vārds, ar kuru pieslēgsies
	\item \texttt{--ssh-password} -- SSH lietotāja parole, to nenorādot, tā tiktu pieprasīta komandas izpildes laikā
	\item \texttt{--use-sudo-password} -- Chef veic darbības kā root lietotājs, tāpēc nepieciešams izmantot lietotāju ar sudo tiesībām
	\item \texttt{--run-list} -- norāda kuras receptes tiks piemērotas pārvaldamajam mezglam
	\item \texttt{--node-name} -- mezgla nosaukums
\end{itemize}

\subsubsection{Lietojot Chef-solo}
Līdzīgi kā Chef-zero, Chef-solo ir vēl viens veids kā piemērot receptes iekārtām bez Chef servera. Chef-solo spēj piemērot receptes, kuras atrodas uz iekārtas.
\begin{itemize}
	\item Uz izstrādātāja darbstacijas lejuplādē visas recepšu grāmatas izmantojot berks vendor
	\item Uzstāda chef-client uz mezgla
	\item Nokopē uzstādāmās recepšu grāmatas uz /var/chef/cookbooks
	\item Izpilda \texttt{chef-solo --override-runlist RUNLIST}
\end{itemize}
Chef-solo komandas --override-runlist karogs nodefinē izpildāmās receptes.

\subsubsection{Hosted Chef}
Strādāt bez Chef servera nav tik viegli. Ar abiem iepriekšminētajiem variantiem ir problēmas, kuras var rasties un var nebūt nemaz tik viegli atrisināmas.
Tāpēc darba izstrādes laikā izmantots \textit{Hosted Chef}, kas ir vienkāršākais veids, kā sākt strādāt ar Chef.
Izveidojot kontu \url{https://manage.chef.io/}, ir iespējams ielogoties grafiskā Chef pārvaldības konsolē.
Pirmais, ko tajā nepieciešams izdarīt ir izveidot jaunu organizāciju. Tad ir iespējams lejuplādēt sākuma paku (angl. \textit{starter kit}), kas ir sagatavots Chef repozitorijs ar nepieciešamo Chef konfigurāciju. Chef servera organizācijās ir arī vairākas noklusējuma grupas (administratori, klienti, lietotāji), kuras tiek izmantotas, lai pārvaldītu izstrādātāju piekļuvi. Kā arī organizācijas satur uzstādītos Chef mezglus. Chef repozitorijā ir slēptā \textit{.chef} mape, kura satur \textit{knife.rb} datni, kurā ir norādīta Chef servera organizācijas adrese.

\section{Mājaslapas izstrāde}
Darba izstrāde tika balstīta uz Agile izstrādes modeli. Sākumā tika noteiktas minimālās "klienta" prasības, veikti mājaslapas un datubāzes struktūres uzmetumi. Notika regulāras tikšanās un prasības tika pastāvīgi papildinātas, gan no "klienta", gan no izstrādātāja puses ieteikumu veidā.

Mājaslapas izstrādē izmantoti šādi rīki:
\begin{itemize}
	\item Ruby on Rails tīmekļa lietojumprogrammu satvars.
	\item CSS -- Bootstrap satvars
	\item JavaScript -- jQuery
	\item Testēšana -- RSpec, Shoulda matchers, Factory Girl, Faker
	\item TravisCI -- CI serviss
	\item Codacy -- statiskas koda analīzes serviss
\end{itemize}

\subsection{Ruby on Rails}
Mājaslapas izstrādē izmantots Ruby on Rails tīmekļa lietojumprogrammu satvars, kas būtiski atvieglo spējīgu mājaslapu izstrādi. Rails


\subsection{RSpec}
RSpec jeb RubySpec ir populārākā Ruby testēšanas bibliotēka. RSpec tika izdots 2005. gadā.
\cite{shayRspec}
RSpec ir BDD testēšanas satvars. RSpec izmanto savu DSL, kas līdzinās dabiskas valodas specifikācijai. Lasot RSpec testus tiem būtu jāizklausās pēc normāliem teikumiem. Tā RSpec cenšas testus padarīt lasāmus un saprotamus.

\subsection{TravisCI}

\subsection{Lietotāji}
\subsection{Uzbrukumi}