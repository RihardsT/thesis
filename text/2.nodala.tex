\chapter{Praktiskais}
\section{Praktiski ar Chef}
Chef strādā pēc vedējsekotājsistēmas principa, tas nozīmē, ka vienmēr ir Chef serveris, kas kalpo kā centrmezgls un pārvalda visus uzstādītos mezglus. Atkarībā no uzņēmuma vajadzībām, Chef serveri ir iespējams uzstādīt vairākos veidos. Chef piedāvā Chef serveri kā pakalpojumu (\textit{Hosted Chef}), kuru ir iespējams izmantot par brīvu līdz 5 mezgliem. Šis ir vienkāršākais un ātrākais veids, kā praktiski sākt strādāt ar Chef, tomēr uzņēmumiem tas varētu būt neatbilstošs, jo nav pilnīga kontrole pār serveri. Ir iespējams Chef serveri uzstādīt \textit{mākonī}. Ir pieejami jau gatavi instanču attēli \textit{Amazon Webservices} un \textit{Microsoft Azure} \textit{mākoņos}, kas būtiski paātrina Chef servera uzstādīšanas laiku. Kā arī ir iespējams Chef serveri uzstādīt uz jebkura servera, uz kura uzstādīta \textit{Red Hat Enterprise Linux} vai \textit{CentOS} 5., 6. vai 7. versija, vai Ubuntu 10.04, 12.04 vai 14.04 versija.
Tā kā darbā uzstādīta tikai viena servera instance izmantojot Chef, darbā izmantots \textit{Hosted Chef} serviss. Tā arī ir paredzēts un ieteicams izmantot Chef.
Apskatīts arī, kā izmantot Chef piedāvātās automatizācijas iespējas bez Chef servera instances, tomēr nevar teikt, ka tas ir ieteicams variants.



\subsection{Darbs ar Chef bez Chef servera}
Ir vairāki varianti, kā piemērot Chef receptes nekontaktējoties ar Chef serveri. Iespējams izmantot chef-zero vai chef-solo.
Izmantojot \textit{chef-zero}, ir iespējams simulēt pagaidu Chef serveri datora atmiņā. Šo vēl vairāk atvieglo Knife-zero spraudnis (angl. \textit{plugin} \url{https://knife-zero.github.io/}), ar kura palīdzību var izdarīt tās pašas darbības, kā tad, ja ir uzstādīts Chef serveris.

Uzstāda chefDK
chef gem install knife-zero
chef generate repo test-repo
cd test-repo
chef generate cookbook cookbooks/testbook -b
cd cookbooks/test-book
berks vendor ../
"-b"  karogs norāda, ka tiks uzstādīts Berkshelf integrācija, kas ļauj ērti atrisināt recepšu grāmatu atkarības.
Izmantojot \textit{berkshelf vendor} komandu, berkshelf lejuplādēs visas recepšu grāmatas, kuras ir nepieciešamas infrastruktūras uzstādīšanai.
knife bootstrap 192.168.1.12 --local-mode --ssh-user vagrant --ssh-password 'vagrant' --sudo --use-sudo-password --node-name chefnode --run-list 'recipe[testbook]'

Lietojot Chef-solo.
Chef-solo spēj piemērot receptes, kuras atrodas uz iekārtas.
Uz izstrādātāja darbstacijas lejuplādē visas recepšu grāmatas izmantojot berks vendor
Uzstāda chef-client uz mezgla
Nokopē uzstādāmās recepšu grāmatas uz /var/chef/cookbooks
Palaiž chef-solo -o RUNLIST
'-o' --override-runlist	- nodefinē izpildāmās receptes

Tomēr strādāt bez Chef servera nav tik viegli. Ar abiem variantiem ir problēmas, kuras var rasties un var nebūt nemaz tik viegli atrisināmas.

Tāpēc darba izstrādes laikā izmantots \textit{Hosted Chef}, kas ir vienkāršākais veids, kā sākt strādāt ar Chef.
Izveidojot kontu \url{https://manage.chef.io/}, ir iespējams ielogoties grafiskā Chef pārvaldības konsolē.
Pirmais, ko tajā nepieciešams izdarīt ir izveidot jaunu organizāciju. Tad ir iespējams lejuplādēt sākuma paku (angl. \textit{starter kit}), kas ir sagatavots Chef repozitorijs ar nepieciešamo Chef konfigurāciju. Chef servera organizācijās ir arī vairākas noklusējuma grupas (administratori, klienti, lietotāji), kuras tiek izmantotas, lai pārvaldītu izstrādātāju piekļuvi. Kā arī organizācijas satur uzstādītos Chef mezglus. Chef repozitorijā ir slēptā \textit{.chef} mape, kura satur \textit{knife.rb} datni, kurā ir norādīta Chef servera organizācijas adrese.

\section{Mājaslapas izstrāde}
Darbā izmantots Ruby on Rails  tīmekļa lietojumprogrammu satvars.
CSS - Bootstrap satvars
JavaScript - jQuery
Testēšana - RSpec, Shoulda matchers, Factory Girl, Faker
RSpec ir uzvedības virzītas izstrādes (angl. \textit{behavior driven development}) testēšanas satvars. RSpec izmanto savu domēnam specifisku valodu, kas līdzinās dabiskas valodas specifikācijai. Lasot RSpec testus tiem būtu jāizklausās pēc normāliem teikumiem. Tā RSpec cenšas testus padarīt lasāmus un saprotamus.
