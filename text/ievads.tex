\chapter*{Ievads}
\addcontentsline{toc}{chapter}{Ievads}

% Cilvēki izsenis ir centušies pasargāt savas telpas no svešu cilvēku piekļuves.

Ventspils Augstskolā ir izveidota elektronikas studentu darba telpa, kurā elektronikas studenti var iet un izstrādāt savus projektus. Šajā telpā atrodas koplietojama elektronikas aparatūra (elektrobarošanas avoti, lodāmuri, u.t.t.). Lai iegūtu piekļuvi studentu telpai, studentam ir jāiziet laikietilpīgs birokrātisks process -- jāiesniedz iesniegums augstskolas vadībā, kur to apstiprina. Šis process ir lēns un var aizņemt pat nedēļu, pēc kā parakstītais iesniegums ir jānodod augstskolas dežurantēm, kuras studentam tad var izsniegt studentu telpas atslēgu.

Kad students vēlas iet strādāt studentu telpā, viņam ir jādodas pie dežurantes palūgt atslēgas, un dežurante pieraksta studenta datus kladē. Tā kā atslēga, ko izsniedz studentiem, ir tikai vienā eksemplārā, rodas gadījumi, ka studentu telpā uzturas vairāki studenti vienlaikus, bet pēc darba atslēgu nodod kāds cits, nevis atslēgas saņēmējs. Tiklīdz viens elektroniķis ir studentu telpā, citi studenti tajā var uzturēties patvaļīgi, kas būtiski apgrūtina telpas uzraudzību. Ja radošā procesa laikā tiek sabojāta kāda no telpas koplietojamajām ierīcēm, studentam vienmēr būtu jāuzņemas atbildība, tomēr praksē tas dažkārt nenotiek. Atbildību nes tas students, kas saņēmis telpas atslēgu, bet gadās, ka atslēgas saņēmējs esošās sistēmas dēļ ne vienmēr atrodas telpā, kas sarežģī telpas uzraudzību.

Pašreizējā sistēma ir neatbilstoša studentu vajadzībām, arī tāpēc, ka studentiem ir nepieciešama piekļuve telpai jebkurā laikā, ko pašreizējā sistēma formāli pieļauj, bet praktiski ir sarežģīti izpildīt. Turklāt šāda arhaiska sistēma pēc būtības ir nedroša, jo cilvēki ir viegli manipulējami un bieži arī neuzmanīgi. Kāds to varētu izmantot, lai nesankcionēti piekļūtu telpai, bojātu un/vai iznestu no telpas tajā esošās ierīces. Dežurantu neuzmanības dēļ atslēga varētu tikt iedota arī studentam, kuram piekļuve telpai nav dota vai būtu jāatņem.

Šī darba ietvaros ir radīta jauna sistēma, kas veiks telpā esošo personu identificēšanu un žurnalēšanu, izmantojot RFID kartes, kā ir atrisinātas vairākas drošības problēmas un neērtības. Izstrādātā sistēma ļautu arī daudz labāk pārvaldīt piekļuvi telpai, kā arī sniegtu iespēju ierobežot piekļuves laikus individuāliem studentiem.
% Ko vēl automatizēt...
% Drošība!!!!!!!!!!!!!!!!!!!!!!!!!!!!!!!!!!!!!!!!!!!!!!!!

Galvenie darba uzdevumi:
\begin{itemize}
  \item Izveidot mājaslapu telpas piekļuves administrēšanai, kas ļautu administratoram autorizēt un pārvaldīt studentu piekļuvi telpai. Studenti varētu sekot līdzi savai piekļuves vēsturei un atsaukt savu piekļuvi telpai, ja gadījies savu RFID kartes nozaudēt.
  \item Uzstādīt mājaslapu uz \textit{RaspberryPi}, kas ir ērts un lēts risinājums vienkāršam serverim.
  \item Imitēt \textit{RFID} karšu lasītāju un savienot to ar serveri, kurš nolasītu studentiem izdalītās RFID kartes un atļautu vai liegtu studentiem piekļuvi telpai.
  % Kartes izdalīs studentiem, karšu tiesības
\end{itemize}

Darbā tiks izmantoti un aprakstīti populāri izstrādes rīki un paņēmieni, kas būtiski atvieglo izstrādi un uzlabo koda kvalitāti.
% Versiju kontrole, testēšana, infrastruktūra kā kods, koda kvalitāte.
% Rīki, kas ļauj abstrahēt ...
Darba praktiskā daļa sadalīta divos programmēšanas uzdevumos:
\begin{env}
  \item Administrēšanas mājaslapas izveide
  \item Infrastruktūras aprakstīšana kodā
\end{env}

% Kaut gan eksistē daudzi komerciāli risinājumi, ne visi no tiem ir droši.

% Ievadam un anotācijām jabūt labai, struktūrai, secinājumiem
