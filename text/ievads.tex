\chapter*{Ievads}
\addcontentsline{toc}{chapter}{Ievads}

% Cilvēki izsenis ir centušies pasargāt savas telpas no svešu cilvēku piekļuves.

Ventspils Augstskolā atrodas studentu telpa, kurā elektronikas studenti var iet un izstrādāt savus projektus. Studentu telpā atrodas dažāda elektronikas aparatūra (elektrobarošanas avoti, lodāmuri, u.t.t.). Tā kā telpā atrodas koplietojama aparatūra, ir nepieciešams, vajadzības gadījumā, saukt studentus pie atbildības, ja viņu radošajā procesā izdodas kādu ierīci saplēst.
% Radošajā procesā lietas tiek saplēstas.
Studentiem nepieciešams iekļūt studentu telpā jebkurā laikā, ko pašreizējā sistēma praktiski nepieļauj.
% Formāli var, praktiski sarežģīti.
% Pašreizējā sistēma.
Lai iekļūtu studentu telpā, studentam ir jādodas pie augstskolas dežurantes un jālūdz atslēgas. Dežurante pieraksta studenta datus kladē. Tā kā atslēga, ko izsniedz studentiem, ir tikai viena, rodas gadījumi, ka uz studentu telpu aiziet vairāki studenti vienlaikus un pēc darba to nodod kāds cits, nevis atslēgas saņēmējs. Tādejādi sarežģījot saukšanu pie atbildības, jo, tiklīdz viens students ir studentu telpā, citi elektroniķi tajā var uzturēties patvaļīgi.
% citiem nav atbildības, tam kam atslēga
% Piemēram studenti grib pa nakti tikt iekšā un nevar
% Esoša ir neatbilstoša vajadzībām
% Izej ārā pačurāt, ctis ieiet un saplēš visu...
Kā arī šāda arhaiska sistēma pēc būtības ir nedroša, jo cilvēki ir viegli manipulējami un bieži arī neuzmanīgi. Kāds to varētu izmantot, lai nesankcionēti piekļūtu telpai. Dežurantu neuzmanības dēļ, atslēga varētu tikt iedota arī studentam, kuram piekļuve telpai nepienākas (nav dota) vai būtu jāatņem.
Automatizēta sistēma ļautu atrisināt šos drošības caurumus un neērtības.

Šajā darbā tiks radīta jauna automatizēta sistēma kas veiks telpā esošo personu identificēšanu un žurnalēšanu.
Piekļuves menedžēšana. Birokrātija - iesniegumi, u.t.t.
% Kaut gan eksistē daudzi komerciāli risinājumi, ne visi no tiem ir droši.

Darba mērķis ir izveidot pilnībā funkcionējošu telpu piekļuves sistēmu, kas ļautu automatizēti žurnalēt un kontrolēt piekļuvi telpai.
% Ko vēl automatizēt...

% Drošība!!!!!!!!!!!!!!!!!!!!!!!!!!!!!!!!!!!!!!!!!!!!!!!!

Galvenie darba uzdevumi:
\begin{itemize}
  \item Izveidot mājaslapu telpas piekļuves administrēšanai
  % Izvērst teikumos un paskaidrot
  \item Uzstādīt mājaslapu uz RaspberryPi
  % Lēts risinājums serverim
  \item Uzstādīt RFID karšu lasītāju un savienot to ar serveri
  % Kartes izdalīs studentiem, karšu tiesības
\end{itemize}

Darbā tiks izmantoti un aprakstīti populāri izstrādes rīki un paņēmieni, kas būtiski atvieglo izstrādi un uzlabo koda kvalitāti.
Versiju kontrole, testēšana, infrastruktūra kā kods, koda kvalitāte.
% Rīki kas ļauj abstrahēt ...


% Ievadam un anotācijām jabūt labai, struktūrai, secinājumiem
