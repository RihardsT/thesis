\chapter*{Ievads}
\addcontentsline{toc}{chapter}{Ievads}

% Cilvēki izsenis ir centušies pasargāt savas telpas no svešu cilvēku piekļuves.

Ventspils Augstskolā atrodas studentu telpa, kurā elektronikas studenti var iet un izstrādāt savus projektus. Studentu telpā atrodas dažāda elektronikas aparatūra (barokļi, lodāmuri, u.t.t.). Tā kā tur telpā atrodas koplietojama aparatūra, ir nepieciešams, vajadzības gadījumā, saukt studentus pie atbildības, ja kaut kas pārstāj darboties.
Studentiem būtu jāspēj iekļūt studentu telpā jebkurā laikā, ko pašreizējā sistēma neļauj.
Lai iekļūtu studentu telpā, studentam ir jādodas pie augstskolas dežurantes un jālūdz atslēgas. Dežurante pieraksta studenta datus kladē. Tā kā atslēga, ko izsniedz studentiem, ir tikai viena, rodas gadījumi, ka uz studentu telpu aiziet vairāki studenti vienlaikus un pēc darba to nodod kāds cits, nevis atslēgas saņēmējs. Tādejādi sarežģījot saukšanu pie atbildības, jo, tiklīdz viens students ir studentu telpā, citi elektroniķi tajā var iet patvaļīgi.
Kā arī šāda manuāla sistēma pēc būtības ir nedroša, jo cilvēki ir viegli manipulējami un bieži arī neuzmanīgi. Kāds to varētu izmantot, lai nesankcionēti piekļūtu telpai. Dežurantu neuzmanības dēļ, atslēga varētu tikt iedota arī studentam, kuram piekļuve telpai nepienākas vai būtu jāatņem.
Automatizēta sistēma ļautu atrisināt šos drošības caurumus un neērtības.

% Kaut gan eksistē daudzi komerciāli risinājumi, ne visi no tiem ir droši.

Darba mērķis ir izveidot pilnībā funkcionējošu telpu piekļuves sistēmu, kas ļautu daudz labāk žurnalēt un kontrolēt piekļuvi telpai.

Galvenie darba uzdevumi:
\begin{itemize}
  \item Izveidot mājaslapu telpas piekļuves administrēšanai
  \item Uzstādīt mājaslapu uz RaspberryPi
  \item Uzstādīt RFID karšu lasītāju un savienot to ar serveri
\end{itemize}

Darbā izmantošu populārus izstrādes rīkus un paņēmienus, kas būtiski atvieglo izstrādi un uzlabo koda kvalitāti.
Versiju kontrole, testēšana, infrastruktūra kā kods, koda kvalitāte.
% Rīki kas ļauj abstrahēt ...
