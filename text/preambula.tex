%Nepieciešamās pakotnes---------------------------------------------------------------------------------------------
\usepackage{polyglossia} 	%Valodu atbalsts
\usepackage{indentfirst} 	%Atkāpes pirmajām rindiņām
\usepackage{titlesec} 	%Virsrakstu noformējumam
\usepackage{graphicx}	%Attēlu importēšanai
\graphicspath{{./images/}}	%Apakšmape bildēm
\usepackage{setspace}	%1.5Rindu atstarpem
\usepackage{layouts}		%textwidth
\usepackage{tabularx}	%Vajadzēs saliktus attēlus
\usepackage{subcaption}	%formulas tekstā
\usepackage{amsmath}
\usepackage{amsfonts}	%\usepackage{amsmath,tabu}
\usepackage[a4paper, lmargin=3.5cm,rmargin=2cm,tmargin=2cm, bmargin=2cm]{geometry}	%Nodefinē attālumus no malām
\usepackage{algorithm}	%Pseidokoda rakstīšanai
\usepackage{algorithmic}
\usepackage{xcolor,colortbl}	%Tabulu šūnu iekrāsošanai
\usepackage{longtable}	%Apjomīgām tabulām
\usepackage{enumitem}	%numeracijai
\usepackage{nomencl}	%Nomenklatūra
\setlist[itemize]{noitemsep, topsep=0pt}
\usepackage{fancyhdr}	%Numerācijai lappuses labajā pusē
\usepackage{natbib}		%Literatūras saraksta veidošanai
\usepackage{hyperref} 	%Garas formulas
\usepackage{amsmath}	%Lai varetu ievietot lapu landscape
\usepackage{lscape}
%Noformējuma parametri---------------------------------------------------------------------------------------------
\setdefaultlanguage{latvian}		%Nodefinējam valodas
\setotherlanguages{english,russian}
%\usepackage[none]{hyphenat}	%vārdu dališanai
\usepackage[title,toc]{appendix}		%Bibliografija
\bibliographystyle{unsrt}
\renewcommand{\setthesubsection}{\arabic{subsection}}
\usepackage{url}	%literatūras saraksta linkiem
%\newcommand{\abstitlestyle}[1]{% Abstract title style
%	\noindent \begin{center}
%		\textbf{\Large #1}
%	\end{center}}
%\usepackage{lmodern}

\newfontfamily{\cyrillicfont}{Times New Roman}
%\setmainfont{Times New Roman}	%Fonts un atstarpe starp rindām
%\newfontfamily\russianfont{Times New Roman}
\fontsize{12pt}{1.5}
\setstretch{1.5}	%Rindiņas pirmā atkāpe
\setlength{\parindent}{1.27cm}
\setlist{leftmargin=2cm}

%Krāsa dažādiem ieēnojumiem
%\definecolor{shadecolor}{rgb}{0.75, 0.85, 0.38}
%Virsrakstu formatējums
%\newfontfamily\sffamily{Times New Roman}
%\newfontfamily\russianfont{Times New Roman}
%\titleformat{\chapter}{\large\centering\sffamily\bfseries}{\thechapter}{0pc}{}
%{\chaptertitlename\ \thechapter}{20pt}{\Huge}

%\newfontfamily\russianfont{Linux Libertine}	% Unicode Computer modern font
%\titleformat{\chapter}{\large\centering\sffamily\bfseries}{\thechapter}{0pc}{}

%\titleformat{\chapter}{\centering\fontsize{16}{16}\bfseries}{\thechapter}{5pt}{}
%\titleformat{\section}{\sffamily\fontsize{14}{14}\bfseries}{\thesection}{5pt}{}
%\titleformat{\subsection}{\sffamily\fontsize{12}{12}\bfseries}{\thesubsection}{5pt}{}

\titleformat{\chapter}{\centering\fontsize{16}{16}\bfseries}{\thechapter}{5pt}{}
\titleformat{\section}{\fontsize{14}{14}\bfseries}{\thesection}{5pt}{}
\titleformat{\subsection}{\fontsize{12}{12}\bfseries}{\thesubsection}{5pt}{}

\titlespacing*{\chapter}{0pt}{10pt}{5pt}
\titlespacing*{\section}{0pt}{10pt}{5pt}
\titlespacing*{\subsection}{0pt}{10pt}{5pt}

%Nodefinējam objektu numerācijas noteikumus
%\def\thechapter{\arabic{chapter}.}
%\def\thesection{\ifx\chapter\undefined{\arabic{section}.}\else  {\thechapter\arabic{section}.}\fi}
%\def\thesubsection{\thesection\arabic{subsection}.}
%\def\thesubsubsection{\thesubsection\arabic{subsubsection}.}
%
%
%\renewcommand{\thefigure}{\arabic{chapter}.\arabic{figure}.}
%\renewcommand{\theequation}{\thechapter \arabic{equation}.}
%\renewcommand{\thetable}{\thechapter \arabic{table}.}
%%Attelu un tabulu nosaukumi
%\captionsetup[figure] {labelformat=default,labelsep=space}
%\captionsetup[table] {labelformat=simple,labelsep=space}
%\renewcommand{\figurename}{att.}

%atstarpe starp formulam un textu
%\setlength{\belowdisplayskip}{0pt}
%\setlength{\belowdisplayshortskip}{0pt}
%\setlength{\abovedisplayskip}{0pt}
%\setlength{\abovedisplayshortskip}{0pt}
%\renewcommand{\thefigure} {\arabic{\figure}}
%\renewcommand{\fnum@figure}{\thefigure ~S \figurename}
%\renewcommand{\tablename}{tabula}
%\makeatletter
%\renewcommand{\fnum@figure}{}
%\makeatother
%Teksta nodaļu fiksētie nosaukumi

\addto\captionslatvian{
\renewcommand\bibname{Izmantotās literatūras un avotu saraksts}
\renewcommand{\contentsname}{Saturs}
\renewcommand{\nomname}{Saīsinājumu un nosacīto apzīmējumu saraksts}
\renewcommand{\appendixtocname}{Pielikumi}%
\renewcommand{\appendixpagename}{Pielikumi}%
}
\renewcommand{\ref}{\nref}
%Nodefine krasu
\definecolor{Gray}{gray}{0.85}
\definecolor{Green}{rgb}{0,1,0}
%Numeracija
\pagestyle{fancy}
\fancyhf{}
\renewcommand{\headrulewidth}{0pt}
\renewcommand{\footrulewidth}{0pt}
\fancyfoot[R]{\thepage}
\fancypagestyle{plain}{%
  %\fancyhf{}%
  \fancyfoot[R]{\thepage}%
}
%Pārnesumiem - ļauj tiasīt lielākas starpas
\hyphenpenalty=5000
%% Atraitņrindiņas un bāreņrindiņas ( widow orphan) vadība
\clubpenalty10000
\widowpenalty10000
% Apakšsvītrām
\usepackage{underscore}
\usepackage{fixlatvian}

\usepackage{caption}
\DeclareCaptionFormat{hfillstart}{\hfill#1#2#3\par}
\captionsetup[table]{format=hfillstart,labelsep=newline,justification=centering, textfont=bf}
\captionsetup[subtable]{position=top, textfont=normalfont, singlelinecheck=off, justification=centering}

\tolerance=1
\emergencystretch=\maxdimen
\hyphenpenalty=10000
\hbadness=10000
